

\subsection{Employment status information}

Prof. Dr. Henrik Leopold, Associate Professor at the Kühne Logistics University, permanent position.\\
Prof. Dr. Han van der Aa, Junior Professor (W1) at the University of Mannheim. Six year contract (ending in March 2026) with an interim evaluation in March 2023.

\subsection{First-time proposal data}

Not applicable.

\subsection{Composition of the project group}

\todo{We obviously need to rewrite this in a smart way. I think we should particularly focus on the complementary aspects. They should not conclude that anyone of us could do this without the other. Highlighting the successful collaboration makes sense obviously.}


The applicants of this proposal have history of successful collaboration. While they share a similar background with respect to the application of natural language processing in process analysis and mining, they both have complementary skills and knowledge. 

\textbf{Prof. Dr. Henrik Leopold} - Henrik Leopold is a tenured Associate Professor at the K\"uhne Logistics University (KLU) and senior researcher at the Hasso Plattner Institute (HPI) at the Digital Engineering Faculty, University of Potsdam. Before joining KLU/HPI in February 2019, he held positions as an Assistant Professor at the Vrije Universiteit Amsterdam (February 2015 – January 2019) and WU Vienna (April 2014 – January 2015) as well as a postdoctoral research fellow at the Humboldt University of Berlin (July 2013 – March 2014). In July 2013, he obtained a PhD degree (Dr. rer. pol.) in Information Systems from the Humboldt University of Berlin. For his thesis he received the TARGION Dissertation Award 2014 for the best doctoral thesis in the field of Information Management and the runner-up of the McKinsey Business Technology Award 2013. Henrik Leopold's research is concerned with leveraging technology from the field of artificial intelligence to develop automated techniques for process analysis and process mining. The results of his research have been published in over 80 publications in books, book chapters, journals, conferences, workshops, and reports. Among others, his research has been published in the journals IEEE Transactions on Knowledge and Data Engineering, IEEE Transactions on Software Engineering, ACM Transactions on Management Information Systems, Decision Support Systems, and Information Systems. 

\textbf{Prof. Dr. Han van der Aa} - Han van der Aa is a Junior Professor (W1) ... \todo{paste your favorite CV here.}

\todo{Conclude by highlighting the complementary aspects.}

\subsection{Researchers in Germany with whom you have agreed to cooperate on this project}
\label{sec:collab:germany}

We agreed to cooperate on this project with two researchers from Germany: \todo{To be filled. My ideas: Heiner or Simone, Matthias, Jana? (we need some diversity)}

%
%\textbf{Prof. Dr. Heiner Stuckenschmidt} - Heiner Stuckenschmidt is Full Professor for Artificial Intelligence and Member of the Data- and Web Science Research Group. He is conducting research in Knowledge Representation and Reasoning as well as their application in semantic information management. He is Co-Editor in Chief of the Journal on Data Semantics and associate editor of the ‘Information Sciences' Journal. He has published more than 200 papers at international peer reviewed conference and journals related to Artificial Intelligence and Information Management.
%
%\textbf{Prof. Dr. Han van der Aa} - Han van der Aa is a Junior Professor in the Data and Web Science Group at the University of Mannheim. Before that, he was an Alexander von Humboldt Fellow, working as a postdoctoral researcher in the Department of Computer Science at the Humboldt-Universität zu Berlin. He obtained a PhD from the Vrije Universiteit Amsterdam in 2018. His research interests include business process modeling, process mining, natural language processing, and complex event processing. His research has been published, among others, in IEEE Transactions on Knowledge and Data Engineering, Decisions Support Systems, and Information Systems.
%
%We have already collaborated with both researchers mentioned above in the context of various research efforts. With Prof. Dr. Heiner Stuckenschmidt, we worked, among others, on process matching using Markov Logic networks \cite{meilicke2017overcoming,leopold2012probabilistic,leopold2015towards}. With Prof. Dr. Han van der Aa, we worked on various topics combining NLP and process analysis \cite{van2017transforming,vanderaa2016comparing,leopold2019using}. Based on these experiences, we are confident that both can provide valuable input for this project. With respect to the work packages, we expect that Prof. Dr. Heiner Stuckenschmidt will provide input for work packages 2 and 3 and that Prof. Dr. Han van der Aa can provide input for work packages 1 and 4. 

\subsection{Researchers abroad with whom you have agreed to cooperate on this project}
\label{sec:collab:abroad}

Outside Germany, we agreed to cooperate with  \todo{To be filled. My ideas: Hajo, Josep, Adela?}
%
%\textbf{Prof. Dr. Hajo A. Reijers} - Hajo Reijers is a Full Professor in the Department of Information and Computing Sciences of Utrecht University, where he holds the chair in Business Process Management and Analytics. He is also a part-time, full professor in the Department of Mathematics and Computer Science of Eindhoven University of Technology, as well as an adjunct professor in the School of Information Systems of Queensland University of Technology (QUT). The focus of his academic research is on business process innovation, process analytics, robotic process automation, and enterprise IT. He published in Information Systems, the Journal of Management Information Systems, the Journal of Information Technology, the International Journal of Cooperative Information Systems, Organization Studies, and Omega, among other journals.
%
%Also with Prof. Dr. Hajo A. Reijers we have already worked on numerous research efforts in the area of process mining \cite{Koorn2020,van2019efficient} and NLP in process analysis \cite{van2017transforming,vanderaa2016comparing,leopold2019using}. What is more, we have worked (and are currently working) together on several research projects funded by the Dutch Research Council (NWO): 
%\begin{itemize}
%\item AutoDrivE - Automatic Derivation of Event Logs  (2019 - 2023)
%\item TACTICS - Techniques for the Analysis of Client-Team Interactions (2017 - 2022)
%\item SADIQ - Software for the Analysis of Dental Implant Quality (2016 - 2017)
%\end{itemize}
%
%Against this background, we believe that the coorporation with Prof. Dr. Hajo A. Reijers will be a valuable addition to this project. 

\subsection{Researchers with whom you have collaborated scientifically within the past three years}

\todo{To be filled.}

%The researchers listed above are also the ones we have mainly collaborated with over the last three years. We plan to use this project to continue and deepen these collaborations. Other research collaborations in the context of process mining and process analysis have been conducted with the following researchers:
%\begin{itemize}
%\item Prof Dr. Avigdor Gal (Israel Institute of Technology): Process mining
%\item Prof. Dr. Jan Mendling (WU Vienna): NLP-based process analysis 
%\item Prof. Dr. Matthias Weidlich (Humboldt University of Berlin): Conformance checking and process matching
%\item Dr. Adela del-R\'{i}o-Ortega (Universidad de Sevilla): Process performance indicators
%\end{itemize}

\subsection{Project-relevant cooperation with commercial enterprises}

We do not plan to cooperate with a commercial enterprise in the context of this project. 
%However, as discussed earlier, we plan to reuse an event log we are currently creating together with the German process mining software vendor Lana Labs\footnote{https://lanalabs.com/} in the research project \textit{AutoDrivE} (Automatic Derivation of Event Logs)\footnote{https://www.nwo.nl/onderzoek-en-resultaten/programmas/open+technologieprogramma/projecten/2018/2018-16672/} funded by the Dutch Research Council (NWO).

%As discussed earlier, we plan to cooperate with the German process mining software vendor Lana Labs\footnote{https://lanalabs.com/}. The main goal of the cooperation is to jointly create an additional data set for the evaluation of the proposed technique. We have already collected experience with Lana Labs as a project partner. In the research project \textit{AutoDrivE} (Automatic Derivation of Event Logs) funded by the Dutch Research Council (NWO), we are cooperating on the topic of automated event log extraction from transactional databases. We are, therefore, confident that also the collaboration in the context of this project will be successful.   

\subsection{Project-relevant participation in commercial enterprises}

We do not have any project-relevant participation in commercial enterprises.  

\subsection{Scientific equipment}

We won't need any further scientific equipment. 

\subsection{Other submissions}

We currently have no other proposal submissions under review.  