

\subsection{Employment status information}

Prof. Dr. Henrik Leopold, Associate Professor at the Kühne Logistics University, permanent position.\\
Prof. Dr. Han van der Aa, Junior Professor (W1) at the University of Mannheim. Six year contract (ending in March 2026) with an interim evaluation in March 2023.

\subsection{First-time proposal data}

Not applicable.

\subsection{Composition of the project group}

\todo{We obviously need to rewrite this in a smart way. I think we should particularly focus on the complementary aspects. They should not conclude that anyone of us could do this without the other. Highlighting the successful collaboration makes sense obviously.}


The applicants of this proposal have history of successful collaboration. While they share a similar background with respect to the application of natural language processing in process analysis and mining, they both have complementary skills and knowledge. 

\textbf{Prof. Dr. Henrik Leopold} - Henrik Leopold is a tenured Associate Professor at the K\"uhne Logistics University (KLU) and senior researcher at the Hasso Plattner Institute (HPI) at the Digital Engineering Faculty, University of Potsdam. Before joining KLU/HPI in February 2019, he held positions as an Assistant Professor at the Vrije Universiteit Amsterdam (February 2015 – January 2019) and WU Vienna (April 2014 – January 2015) as well as a postdoctoral research fellow at the Humboldt University of Berlin (July 2013 – March 2014). In July 2013, he obtained a PhD degree (Dr. rer. pol.) in Information Systems from the Humboldt University of Berlin. For his thesis he received the TARGION Dissertation Award 2014 for the best doctoral thesis in the field of Information Management and the runner-up of the McKinsey Business Technology Award 2013. Henrik Leopold's research is concerned with leveraging technology from the field of artificial intelligence to develop automated techniques for process analysis and process mining. The results of his research have been published in over 80 publications in books, book chapters, journals, conferences, workshops, and reports. Among others, his research has been published in the journals IEEE Transactions on Knowledge and Data Engineering, IEEE Transactions on Software Engineering, ACM Transactions on Management Information Systems, Decision Support Systems, and Information Systems. 

\textbf{Prof. Dr. Han van der Aa} - Han van der Aa is a Junior Professor (W1) in School of Business Informatics at the University of Mannheim, where he heads the research group on process analytics since April 2020. Before that, he was an Alexander von Humboldt Fellow, working as a postdoctoral researcher in the Department of Computer Science at the Humboldt-Universität zu Berlin (May 2018 - March 2020). He obtained a PhD in computer science from the Vrije Universiteit Amsterdam in January 2018. His research interests include business process modeling, process mining, natural language processing, and complex event processing. 
The results of his research have so far resulted in close 50 publications, including articles in 
renowned international journals such as IEEE TKDE, DSS, and Information Systems, as well as at the CAISE, BPM, SIGMOD, and COLING conferences.
He is a PC member of established conferences such as BPM, ICPM, and DEBS.

\todo{Conclude by highlighting the complementary aspects.}

\subsection{Researchers in Germany with whom you have agreed to cooperate on this project}
\label{sec:collab:germany}

We agreed to cooperate on this project with three researchers from Germany: Prof. Dr. Stefanie Rinderle-Ma, Prof. Dr. Matthias Weidlich, Prof. Dr. Simone Ponzetto. 

\textbf{Prof. Dr. Stefanie Rinderle-Ma}: 

 
\textbf{Prof. Dr. Matthias Weidlich}: Matthias Weidlich is a full professor at the Department of Computer Science at Humboldt- Universi\"at zu Berlin and has been an Emmy Noether Research group leader at the same institute. Before that, he held positions at the Department of Computing at Imperial College London and at the Technion - Israel Institute of Technology. He holds a PhD from the Hasso Plattner Institute (HPI), University of Potsdam. His research focuses on process-oriented and event-driven systems and his results appear regularly in the premier conferences (VLDB, SIGMOD) and journals (TKDE, Inf. Sys., VLDBJ) in the field.

\textbf{Prof. Dr. Simone Ponzetto}: Simone Ponzetto is a professor of Information Systems at the University of Mannheim, where he leads the Natural Language Processing (NLP) and Information Retrieval group. His main research interests lie in the areas of knowledge acquisition, text understanding, and the application of NLP methods for research in the (digital) humanities and (computational) social sciences. Simone regularly serves as area chair and program committee member of *ACL and (IJC/AA)AI conferences and is an editorial board member of the Artificial Intelligence Journal and Journal of Natural Language Engineering. He is (co-)author and (co-)editor of over 100 refereed papers in scientific journals, books, and conference proceedings.

We have already collaborated with all three researchers mentioned above in the context of various research efforts. 
With Prof. Dr. Stefanie Rinderle-Ma, we worked, among others, on extracting process information from textual resources \cite{winter2020assessing}. 
With Prof. Dr. Matthias Weidlich, we worked various topics related to process mining \cite{} and low level event data \cite{}
With Prof. Dr. Simone Ponzetto, we have on ongoing cooperation on extraction of process information from text.

Based on these experiences, we are confident that both can provide valuable input for this project. With respect to the work packages, we expect \todo{finish}.


\subsection{Researchers abroad with whom you have agreed to cooperate on this project}
\label{sec:collab:abroad}

Outside Germany, we agreed to cooperate with Prof. Dr. Josep Carmona, Dr. Chiara Ghidini

 \todo{To be filled.}

\textbf{Prof. Dr. Josep Carmona}: Josep Carmona is an Associate Professor at Universitat Politècnica de Catalunya (UPC). He received a PhD at the same university in 2004, under the supervision of Prof. Jordi Cortadella. His research interests include formal methods and concurrent systems, data and process science, business intelligence and business process management, and natural language processing. He has co-authored numerous research papers and organized various conferences and workshops. He is a founder of the International Conference on Process Mining, where he is part of the Steering Committee, and a member of the IEEE Task Force on Process Mining.

%Prof.\ Carmona has conducted various works on the intersection of NLP and process analysis, including a collaboration with the applicant on the comparison of textual process descriptions to process models~\cite{sanchez2018aligning}. 


%
%\textbf{Prof. Dr. Hajo A. Reijers} - Hajo Reijers is a Full Professor in the Department of Information and Computing Sciences of Utrecht University, where he holds the chair in Business Process Management and Analytics. He is also a part-time, full professor in the Department of Mathematics and Computer Science of Eindhoven University of Technology, as well as an adjunct professor in the School of Information Systems of Queensland University of Technology (QUT). The focus of his academic research is on business process innovation, process analytics, robotic process automation, and enterprise IT. He published in Information Systems, the Journal of Management Information Systems, the Journal of Information Technology, the International Journal of Cooperative Information Systems, Organization Studies, and Omega, among other journals.
%
%Also with Prof. Dr. Hajo A. Reijers we have already worked on numerous research efforts in the area of process mining \cite{Koorn2020,van2019efficient} and NLP in process analysis \cite{van2017transforming,vanderaa2016comparing,leopold2019using}. What is more, we have worked (and are currently working) together on several research projects funded by the Dutch Research Council (NWO): 
%\begin{itemize}
%\item AutoDrivE - Automatic Derivation of Event Logs  (2019 - 2023)
%\item TACTICS - Techniques for the Analysis of Client-Team Interactions (2017 - 2022)
%\item SADIQ - Software for the Analysis of Dental Implant Quality (2016 - 2017)
%\end{itemize}
%
%Against this background, we believe that the coorporation with Prof. Dr. Hajo A. Reijers will be a valuable addition to this project. 

\subsection{Researchers with whom you have collaborated scientifically within the past three years}

\todo{To be filled.}

The researchers listed above are also the ones we have mainly collaborated with over the last three years. We plan to use this project to continue and deepen these collaborations. Other research collaborations in the context of process mining, process analysis, and natural language processing have been conducted with the following researchers:

Han:

\begin{itemize}
\item Prof Dr. Avigdor Gal (Israel Institute of Technology): Process mining
\item Prof. Dr. Jan Mendling (WU Vienna): NLP-based process analysis 
\item Prof. Dr. Heiner Stuckenschmidt 
\item Farbricio  Maggi:
\item Adela
%\item Prof. Dr. Matthias Weidlich (Humboldt University of Berlin): Conformance checking and process matching
%\item Dr. Adela del-R\'{i}o-Ortega (Universidad de Sevilla): Process performance indicators
\end{itemize}

Henrik: Hajo, Flavia, Weske, 


\subsection{Project-relevant cooperation with commercial enterprises}

We do not plan to cooperate with a commercial enterprise in the context of this project. 
%However, as discussed earlier, we plan to reuse an event log we are currently creating together with the German process mining software vendor Lana Labs\footnote{https://lanalabs.com/} in the research project \textit{AutoDrivE} (Automatic Derivation of Event Logs)\footnote{https://www.nwo.nl/onderzoek-en-resultaten/programmas/open+technologieprogramma/projecten/2018/2018-16672/} funded by the Dutch Research Council (NWO).

%As discussed earlier, we plan to cooperate with the German process mining software vendor Lana Labs\footnote{https://lanalabs.com/}. The main goal of the cooperation is to jointly create an additional data set for the evaluation of the proposed technique. We have already collected experience with Lana Labs as a project partner. In the research project \textit{AutoDrivE} (Automatic Derivation of Event Logs) funded by the Dutch Research Council (NWO), we are cooperating on the topic of automated event log extraction from transactional databases. We are, therefore, confident that also the collaboration in the context of this project will be successful.   

\subsection{Project-relevant participation in commercial enterprises}

We do not have any project-relevant participation in commercial enterprises.  

\subsection{Scientific equipment}

We won't need any further scientific equipment. 

\subsection{Other submissions}

We currently have no other proposal submissions under review.  