 \begin{itemize}
\item 1 - 1.5 pages general intro / positioning / and challenges
\item figure should make clear what we want to achieve
\item Story: Process mining is widely used to discover, analyze, and improve processes. Traditional process mining techniques reconstruct how a process is executed by analyzing so-called event logs. These event logs are extracted from IT systems and, therefore, provide insights into what exactly happened in the organization. A key assumption, however, is that the events in these event logs have a certain level of granularity, i.e. , they relate to activities such as analyzing an order or sending an invoice. If the events relate to very fine-grained actions, such as clicking on a button or entering a name into a text field, traditional process mining techniques are not very helpful... 
\end{itemize}


\begin{figure}[h!]
\centering
\caption{Super cool example}
\label{fig:example}
\end{figure}

Figure \ref{fig:example} illustrates this idea. It shows ... 

These examples illustrate that ...  However, to automatically achieve this, X main challenges need to be overcome: 

\begin{itemize}
\item \textit{Challenge 1}: 
\item ...
\end{itemize}

An effective technique ... should be able to successfully cope with all these challenges. In the following, we review the state of the art and show that these challenges are not adequately handled up to now.