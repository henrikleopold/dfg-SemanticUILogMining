\label{sec:startingpoint}

Process mining is widely used to discover, analyze, and improve business processes 
by analyzing event data extracted from IT system, stored in so-called event logs~\cite{van2016data}.
A key task in this regard is \emph{process discovery}, which aims to reconstruct how a process was truly executed. To do so, process discovery strives to establish an accurate process model on the basis of the recorded behavior captured in an event log.
Using such event logs as basis for discovery has an important limitation, however: It limits the scope of analysis to \textit{back-end events}, i.e., secondary, indirect events that were triggered by the actual user activity~\cite{diba2020extraction}. User activities that do not result in such back-end events or take place in productivity applications such as Excel and Outlook, are thus not recorded in event logs and, therefore, invisible to traditional process mining and discovery techniques. 

To avoid this problem and be able to obtain a comprehensive view on business processes, the goal of this proposal is to enable
%develop a novel technique that conducts 
process discovery based on \textit{user interaction (UI) logs} rather than traditional event logs. In essence, an UI logs is a collection of recorded interactions performed on GUI components, such as clicks on buttons or keyboard entries in text areas \cite{Urabe21}. The benefit of this is that UI logs
can be obtained for any business process of which the activities are performed on a computer, regardless of the specific software applications required for it.
%regardless of the software applications that are used for its execution, as long as  The only requirement is that relevant activities are performed using a computer. 
Available \textit{logging software} is then able to extract and store relevant data such as the interaction type (e.g. click or keyboard stroke), the interaction time, and the interaction context (e.g., the affected GUI element, window title, URL, etc.) in an UI log  \cite{leno2019action}, of which 
\autoref{fig:example} shows a simplified excerpt.
%Figure \ref{fig:example} shows a simplified excerpt of such an UI log. 
The events from this UI log show how a user receives orders via e-mail (cf., events 1 to 3) and proceeds to handle them in Salesforce's web application (cf., events 4 to 7).

%handles order requests using the application Salesforce that are received via e-mail. 

\begin{figure}[h!]
\centering
 \begin{adjustbox}{max width=\textwidth}
\begin{tabular}{llllllll}
\hline\noalign{\smallskip}\noalign{\smallskip}
\textbf{ID} &\textbf{Timestamp}&\textbf{Event}&\textbf{Application}&\textbf{Element label}&\textbf{Element type}&\textbf{Element value}&\textbf{URL}\\
\noalign{\smallskip}\hline\noalign{\smallskip}
1&08:35.2&click&Outlook&Customer X - O123&list&Please initiate an order …&-\\\noalign{\smallskip}
2&08:35.2&click&Outlook&Customer X - O234&list&Please initiate an order …&-\\\noalign{\smallskip}
3&08:35.2&click&Outlook&Customer Y - O789&list&Please initiate an order …&-\\\noalign{\smallskip}
4&08:39.7&click&Chrome&Log in&button&-&https://www.salesforce.com/\\\noalign{\smallskip}
5&08:40.0&change&Chrome&Password&text field&-&https://login.salesforce.com/\\\noalign{\smallskip}
6&08:40.5&click&Chrome&Submit&button&-&https://login.salesforce.com/\\\noalign{\smallskip}
7&08:52.6&click&Chrome&New Account&button&-&https://com.lightning.force.com/home\\\noalign{\smallskip}
8&08:53.2&change&Chrome&New Order&text field&Customer X&https://com.lightning.force.com/acc/\\\noalign{\smallskip}
9&08:53.9&ctrl + c&Outlook &Customer X - O123&list&Please initiate an order …&-\\\noalign{\smallskip}
10&08:54.3&click&Chrome&Billing address&text field&-&https://com.lightning.force.com/acc/\\\noalign{\smallskip}
11&08:54.4&ctrl + v&Chrome&Billing address&text field&Hofstraße 14, ... &https://com.lightning.force.com/acc/\\\noalign{\smallskip}
12&08:54.9&click&Chrome&Save&button&-&https://com.lightning.force.com/acc/\\\noalign{\smallskip}
13&08:40.0&change&Chrome&Password&text field&-&https://www.facebook.com/\\\noalign{\smallskip}
14&08:42.9&click&Chrome&Log in&button&-&https://www.facebook.com/\\\noalign{\smallskip}
15&08:42.9&click&Chrome&Messenger&button&-&https://www.facebook.com/\\\noalign{\smallskip}
16&08:44.1&click&Chrome&New message&list&Hey, how are you? …&https://www.facebook.com/\\\noalign{\smallskip}
17&08:56.7&click&Outlook&Customer X - O234&list&Please initiate an order …&-\\\noalign{\smallskip}
18&08:58.2&change&Chrome&New Order&text field&Customer X&https://com.lightning.force.com/acc/\\\noalign{\smallskip}
19&08:58.6&click&Chrome&Upload files&button&CustomerX-2021-O234.docx&https://com.lightning.force.com/acc/\\\noalign{\smallskip}
\hline\noalign{\smallskip}
\end{tabular}
\end{adjustbox}
\caption{Simplified excerpt of an UI log}
\label{fig:example}
\end{figure}

Conducting process discovery based on such UI logs is a complex task, though.
%A closer look at the UI log from \autoref{fig:example} reveals that conducting process discovery on such a log is a complex task. 
A key issue is that a log such as the one in \autoref{fig:example}, fails to fulfill basic requirements for process mining~\cite{van2016data}: it does not provide clear event labels that indicate to which process step an event corresponds (e.g., \emph{receive order} for event 1), it contains events that are not relevant to the process under analysis 
(cf., events 13 to 16), and  it does not contain a case identifier that allows us to 
group together events related to the same process instance. Furthermore, even if such problems are overcome, UI logs record events at a very low level, meaning that process models directly discovered from them will be too large and complex to obtain any useful insights into a process's execution. Against this background, we define two main problem areas for our project, each coming with specific challenges. 

\mypar{Data transformation} We need to transform the raw UI log data into an event log that fulfills the above-mentioned requirements. This comes with three main challenges:

\noindent \textit{Event annotation}: Since UI logs do not contain event labels revealing what precisely happened, we need to automatically annotate the events with relevant context information. To illustrate this, consider event 1 from \autoref{fig:example}. In a traditional event log, this event would probably carry a label such as ``\textit{Receive order}''. In the present UI log, we see that the event was a ``\textit{click}'' on a ``\textit{list}'' in the application ``\textit{Outlook}''. That this click relates to receiving an order from a customer can only be inferred from the associated e-mail. Where such information can be inferred will, however, differ from event to event. 

\noindent \textit{Noise filtering}: UI logs may contain events that do not relate to the actual process execution, so-called noise. Examples include visits to social media platforms, checking private e-mails, and ordering private items in online shops. In the UI log from \autoref{fig:example}, we observe an example from the first category (see events 13 to 16). The automatic recognition of noise is complex since not all noisy events can be recognized via the type of application or the URL. As an example consider reading a private e-mail that was sent to the professional e-mail address of the user.

\noindent \textit{Case identification}: The notion of \textit{case identifier} is generally missing in UI logs \cite{leno2021robotic}. To illustrate the challenge of introducing a case identifier, again consider \autoref{fig:example}. From the overall context, we can then infer that events 4 to 12 relate to a case concerned with order \textit{O123} from Customer X. This, however, can be only safely inferred from event 11 where the billing address from Customer X is entered into the system. Events 17 to 19 then relate to a case concerned with order \textit{O234} from Customer X. This, however, we can only indirectly infer from the name of the file that is uploaded to Saleforce. 

\mypar{Process representation} We need to provide a proper representation of the UI log that visualizes the process at an appropriate level of granularity such that users can obtain the insights they require. This again comes with three main challenges: 

\noindent \textit{Event abstraction}: A process model that is directly derived from the events from an UI log will be very large and only contain low level events such as ``\textit{Click confirm button}'' or ``\textit{Open e-mail}". From an analytical point of view, however,  it would be desirable to include higher level events that represent actual business activities such as ``\textit{Create order}'' or ``\textit{Contact customer}''. This process of event abstraction is highly challenging since UI events constituting a business activity may not be executed consecutively and in varying ways. As an example, consider the events relating to the creation of orders \textit{O123} and \textit{O234}. The creation of the first order involves entering a billing address while the creation of the second order involves uploading a file. 

\noindent \textit{Event labeling}: A useful process representation requires clear and expressive text labels. Therefore, once higher-level events have been identified, they still need to be labeled properly. Automatically concluding that, for instance, events 8 to 12 can be appropriately described by ``\textit{Create order}'' is considerably complex. Here it is required to recognize the overall context and the final outcome of a series of low level events.  
 
\noindent \textit{Model discovery}: The outcome of process discovery is typically a process model-based representation. However, existing process discovery algorithms have not been developed to properly represent low level events from UI logs. The challenge, therefore, is to develop an technique that can provide the users with an effective representation of the process from the UI log. 

 
%\mypar{Data transformation} The data transformation comes with three main challenges: event annotation, noise filtering, and case identification. 
%
%UI logs do not contain event labels revealing what precisely happened. Hence, we need to \textit{annotate the events}. To illustrate this, consider event 1 from \autoref{fig:example}. In a traditional event log, this event would probably carry a label such as ``\textit{Receive order}''. In the present UI log, we see that the event was a ``\textit{click}'' on a ``\textit{list}'' in the application ``\textit{Outlook}''. That this click relates to receiving an order from a customer can only be inferred from the associated e-mail. Events 4 and 14 also highlight the importance of proper event labels for classifying events. Looking at the key attributes \textit{Event}, \textit{Application}, \textit{Element label}, and \textit{Element type}, they seem to be identical. However, in fact, event 4 leads the user to a log in screen (the password entry succeeds the event), while event 14 completes the log in process (the password entry precedes the event). Proper labels could have clarified this difference. Here, only the \textit{URL} attribute helps to recognize that the specific application context differs. Unfortunately, the application in which an event occurred is sometimes hard to determine. For example, consider events 4, 7, and 13. According to the UI log, these events all occurred in the context of the application ``\textit{Chrome}'', i.e., an Internet browser. However, a brief analysis of the respective URL attributes reveals that event 4 relates to Salesforce and event 13 relates to Facebook. The fact that also event 7 relates to Salesforce is actually hard to identify since the URL structure of Salesforce changes once the user has logged into the application.

%As already pointed out, UI logs may contain events that do not relate to the actual process execution, so-called \textit{noise}. Examples include visits to social media platforms, checking private e-mails, and ordering private items in online shops. In the UI log from \autoref{fig:example}, we observe a case from the first category: The user briefly switches to the social media platform Facebook to read a message before resuming the work in Salesforce. Note that not all noisy events can be recognized via the type of application or the URL. As an example consider reading a private e-mail that was sent to the professional e-mail address of the user. All these events must be properly recognized and removed since they are not part of the actual process execution. 
  
%The notion of \textit{case identifier} is generally missing in UI logs \cite{leno2021robotic}. To illustrate the challenge of recognizing which events in an UI log belong to the same case, again consider the events from \autoref{fig:example}. In the beginning of the log, we observe three events that potentially relate to three different cases. It seems that the user checks more than a single e-mail before picking up the first order. From the overall context, we can then infer that events 4 to 12 relate to the case concerned with order \textit{O123} from Customer X. This, however, can be only safely inferred from event 11 where the billing address from Customer X is entered into the system. Events 17 to 19 then relate to the case concerned with order \textit{O234} from Customer X, which we first encountered in event 2. Again, we can only indirectly infer this from the name of the file that is uploaded to Saleforce. 



An effective technique for discovering process models from UI logs should be able to successfully cope with all these challenges. In the following, we review the state of the art and show that these challenges are not adequately handled up to now. \todo{(We need to briefly talk about the HOW already, i.e., that we will use NLP.)}