 
\subsubsection{State of the art} 
 \label{sec:stateoftheart}
 
The proposed project primarily relates to research on traditional process mining (using event logs) and robotic process mining (using UI Logs). Below, we briefly review these streams and highlight the gaps that exist with respect to the challenges identified above.  

\mypar{Process mining} Process mining is a family of data analysis techniques that facilitate the discovery, analysis, and improvement of business processes \cite{van2016data}. The core idea of process mining techniques is to analyze so-called \textit{event logs}. These event logs are extracted from information systems that support the execution of business processes and, therefore, capture how these processes are actually executed. Available process mining techniques serve a wide range of tasks including process discovery, conformance checking, and enhancement. Techniques for \textit{process discovery} (see e.g. \cite{gunther2007fuzzy,weijters2011flexible,leemans2013discovering}) aim to provide the user with a visual representation of the process captured in the event log. \textit{Conformance checking} techniques (see e.g. \cite{rozinat2008conformance,adriansyah2011conformance}) detect differences between the actual and intended process execution by comparing the event log with a normative process model. Techniques for \textit{enhancement} again address a variety of tasks such as predicting relevant aspects of the process execution \cite{di2018predictive} or repairing a given process model based on the event log \cite{polyvyanyy2016impact}. Many of the challenges highlighted in the two problem areas above, also play a role in these ``traditional'' facets of process mining. Specifically, process mining techniques have been concerned with the challenges of noise filtering and case identification (problem area 1) as well as the challenges of event abstraction, event labeling, and model discovery (problem area 2):
\vspace{0.2em}
\newline%
\noindent \textit{Noise filtering:} The removal of noise directly affects the quality of process models generated by process discovery techniques, as well as of other process mining results. 
Therefore, various discovery techniques take noise into account explicitly (e.g.,~\cite{weijters2003rediscovering,leemans2013discovering,van2016avoiding}), whereas there are also dedicated techniques available  for removing noise from events logs~\cite{tax2017discovering,CHENG2015138}. However, one of the key assumptions of all these techniques is that noise is highly infrequent. This is problematic in our context, since if certain process-irrelevant actions (e.g., visiting social media websites) occur on a regular basis, they would not be recognized as noise by frequency-based techniques. 
\vspace{0.2em}
\newline%
\noindent \textit{Case identification:}  The problem of case identification in event logs is addressed by various existing techniques (cf.,~\cite{diba2020extraction} for an overview). However, they consider rather restrictive settings. The technique from Ferreira and Gillblad ~\cite{ferreira2009discovering} provide a solution for processes that do not contain loops or activity repetitions, whereas the technique from Bayomie et al.~\cite{bayomie2019probabilistic} assumes that a process model is already available. Both are assumptions that will not be met in the context of our project. 
\vspace{0.2em}
\newline%
\noindent \textit{Event abstraction:} 
The issue of event abstraction has also been discussed in the context of traditional process mining~\cite{van2020event,diba2020extraction}. Recognizing that recorded events can differ widely in their granularity, several abstraction techniques have been proposed to obtain consistent and useful event logs or process models (e.g. \cite{baier2014bridging,van2020event,de2020event}). Available techniques differ with respect to many aspects such as the type of supervision (supervised/ unsupervised), the handling of concurrency (yes/ no), and the type of output (probabilistic/ deterministic). What is currently still missing from the perspective of this project is an unsupervised technique that can can deal with the large degree of variability in UI logs and can reliably recognize respective higher-level business activities. 
\vspace{0.2em}
\newline%
\noindent \textit{Event labeling:} 
While the problem of labeling higher-level activities has been recognized and discussed in the context of process mining~\cite{van2020event,van2016enabling}, no dedicated techniques to overcome this problem currently exist. Instead, the  currently proposed solution is to delegate this task to domain experts. For our goal of automated discovery of processes, this caveat must thus be addressed.   
\vspace{0.2em}
\newline%
\noindent \textit{Model discovery:} 
While various process discovery techniques exist (cf., \cite{augusto2018automated} for an overview), the vast majority are designed to deal with higher-level event logs, rather than the low-level data found in UI logs. Closest to our goal is recent work on multi-level process discovery~\cite{leemans2020using}, which is the first to recognize the importance of granularity differences in discovery. Yet, the process models it yields lack intuitiveness, whereas it also imposes strong requirements on the input it can handle. As such, it provides a useful foundation for our project, but leaves a considerable gap to be addressed.

%In principle, each of the available process model discovery techniques could be used to address the model discovery challenge of this proposal. However, existing discovery techniques were designed for traditional event logs and not for UI logs. Therefore, they do not provide dedicated mechanisms to deal with the problem of granularity.  
 
% \noindent\fbox{%
%\parbox{0.985\textwidth}{%
%In summary, traditional process mining research has recognized five of the six challenges we identified for this project. Available solutions, however, are not applicable because of a limited scope, simplifying assumptions, or an insufficient degree of automation.
%}}

\mypar{Robotic process mining} Robotic process automation (RPA) is a technology that aims to automate repetitive human work. The core idea is to let software robots (or bots) mimic the actions of a human directly in a GUI \cite{SYED2020103162}. A key requirement of RPA is to actually identify automatable tasks.  Recognizing this, the research domain of robotic process mining (RPM) has emerged in recent years \cite{leno2021robotic}. The goal of RPM techniques is to automatically identify automatable routines based on UI logs. 
By doing so, RPM faces several challenges with the ones identified for the proposed project, primarily with respect to noise filtering and case identification from problem area 1:   
\vspace{0.2em}
\newline%
\noindent \textit{Noise filtering:} 
Also in the context of RPA, noisy events, such as social media visits, need to be removed from UI logs since they should not become part of automated procedures. However, effective solutions for noise removal from UI logs are missing. While noise removal techniques from traditional process mining (see above) are generally applicable, their limitations for removing noise from UI logs have also been recognized \cite{leno2021robotic}. 
As a solution, some authors propose supervised noise removal based on an existing process model \cite{agostinelli202111} or they suggest using rules \cite{bosco2019discovering,leno2020identifying}. However, such process models cannot be expected to be available in the context of our project, whereas rule-based approaches are too rigid to deal with the flexibility of real-world UI logs.
\vspace{0.2em}
\newline%
\noindent \textit{Case identification:} The identification of cases in RPM conceptually differs from case identification in traditional process mining. The underlying assumption in RPM is that cases do not overlap and, thus, that case identification can be achieved by splitting the log into segments. 
Researchers have proposed manual, supervised, and unsupervised approaches for this segmentation task. Urabe et al.~\cite{urabe2019visualizing} introduced a manual approach that visualizes the UI log using a graph and, in this way, supports the user in identifying segment boundaries. Agostinelli et al. \cite{agostinelli202111} proposed a supervised approach, requiring a process model, that leverages trace alignments from conformance checking. 
%This approach, however, requires a Petri net representing the underlying process as input. 
Unsupervised approaches were introduced by Leno et al. \cite{leno2020identifying,leno2022discovering}. They construct a control-flow graph from the UI log and use back edges detection to identify segment boundaries. Another unsupervised approach from Urabe et al. \cite{Urabe21} leverages the concept of co-occurrence from topic segmentation in natural language processing to segment the UI log. Despite the potential of these techniques, 
the assumption that cases are executed in a strictly sequential manner does not hold for many real-world settings, in which users may work concurrently on different cases (cf., \autoref{fig:example}). Naturally, there also hybrid approaches available that combine supervised and unsupervised methods with human feedback. In a recent paper, Agostinelli et al. propose such a hybrid approach relying on  frequent-pattern identification and human-in-the-loop interaction~\cite{agostinelli2022mastering}. 



\subsubsection{Preliminary work}
\label{sec:preliminarywork}

The applicants, Prof.\ Leopold and Prof.\ Van der Aa, have published various works that relate to the proposed project. Both applicants have considerable expertise when its comes to the use of natural language processing (NLP) for the purposes of process analysis through individual as well as joint works, which forms a key component in the proposed solutions.
Furthermore, the project relates to individual experience of the applicants with respect to the analysis of low-level event data (Prof.\ Van der Aa) and to the understandability of process representations (Prof.\ Leopold):


\mypar{NLP for process analysis}
The applicants have developed approaches for the extraction of semantic information, such as actions and business objects, from activity labels in process models~\cite{leopold2013detection,leopold2019using} and data attributes in event logs~\cite{rebmann2021extracting}.
This expertise shall provide the foundation for event annotation in UI logs, a primary task in the proposed project. A key distinction, though, is that the existing works are designed to deal with short fragments (such as ``\emph{Create purchase order}''), whereas the project at hand shall also deal with larger texts, such as e-mails. In that regards the applicants can also build on their expertise when extracting process information from textual process descriptions, e.g., for the recognition of process constraints~\cite{van2019extracting,winter2020assessing} and for querying~\cite{leopold2019searching}. Here, a key distinction is that those approaches were designed for process-oriented texts, whereas the proposed project shall also deal with texts that are not structured in this manner.

Recently, the applicants showed the potential of employing semantic information in process mining when using extracted business objects and actions for the purposes of \emph{semantic anomaly detection}~\cite{van2021natural}. Specifically, the work demonstrates that NLP can be leveraged to detect process behavior that violates commonsense rules. This provides a novel angle for anomaly detection in comparison to existing works, which deem process behavior to be anomalous when it is infrequent.
This work shall provide a starting point for the noise filtering that is required in the proposed project. However, next to the recognition of behavioral anomalies, the project also requires techniques that are specifically designed to recognize and remove non-business related events.




\mypar{Analysis of low-level event data} 
Prof.\ Van der Aa has worked on several projects involving the analysis of low-level event data in both event logs and streams, which have clear similarities to the data granularity used in the proposed project.
Existing work relates to the identification of key event patterns in streams~\cite{vanderaa2021cep}, which can provide a basis for the recognition of behavioral regularities for case identification.
Furthermore, experience on the efficient handling of large amounts of low-level events~\cite{zhao2021eires} can aid the computational efficiency of approaches developed for the proposed project, especially for tasks such as case identification and log abstraction that require global optimization.
Finally, ongoing work on log abstraction with guarantees~\cite{rebmann2021icdesubm} can be used as a starting point for the meaningful abstraction of UI logs, since the characteristics of these logs can be used to guide the abstraction task, e.g., by ensuring that events from different systems are not grouped together.

\mypar{Understandability of processes}
Prof.\ Leopold has worked extensively on the topic of process model understandability, especially from a linguistic angle. Among others, he has developed techniques for automatically recognizing and correcting linguistic problems in process models that negatively affect the understandably of process models~\cite{leopold2013detection,leopold2012refactoring,pittke2015automatic}. In this context, he has also proposed a technique that attempts to automatically determine names for (to be aggregated) process model fragments \cite{leopold2014simplifying}.  The challenge of labeling events that result from event abstraction in event logs is conceptually highly similar to labeling activities that result from activity abstraction in process models. However, the technique from \cite{leopold2014simplifying} relies on the specifics of event-driven process chains (EPCs) and, therefore, cannot be transferred to labeling higher-level events. Nonetheless, these works provide important input for the challenges of problem area 2. The technical aspects represent an important basis for the challenges of event abstraction and activity labeling. The experience collected with user experiments, for instance in \cite{pittke2015automatic}, will help us to successfully complete the challenge of process discovery and representation. 
