%!TEX program = xelatex

\documentclass[a4paper,10pt,oneside]{book}
\usepackage[utf8]{inputenc}
\usepackage[usenames,svgnames]{xcolor} % Allows the definition and use of custom colours
\usepackage{graphicx}
\usepackage{datetime}
\usepackage{pdfpages}
\usepackage[hidelinks]{hyperref}
\usepackage{footnote}
\usepackage{booktabs}
\usepackage{datetime}
\usepackage{paralist}
\usepackage{xspace}
\usepackage{xstring}
\usepackage{titletoc}

\input{structure.tex} % Include the file that specifies the document structure

% preamble of cover letter
\usepackage{lmodern}
\renewcommand*\familydefault{\sfdefault}
\usepackage[T1]{fontenc}

% Set Page layout:
%\usepackage{changepage}
%\changepage{text height}{text width}{even-side margin}
%{odd-side margin}{column sep.}
%{topmargin}{headheight}{headsep}{footskip}
%\changepage{+3cm}{}{}{}{}{}{}{}{-5cm}

\newdateformat{mydateformat}{\THEDAY~\monthname[\THEMONTH] \THEYEAR}
\newdateformat{mydateformattwo}{\monthname[\THEMONTH] \THEYEAR}


\usepackage{etoolbox}
\makeatletter
\patchcmd{\chapter}{\if@openright\cleardoublepage\else\clearpage\fi}{}{}{}
\patchcmd{\part}{\if@openright\cleardoublepage\else\clearpage\fi}{}{}{}
\makeatother

\newcommand{\myheader}[1]{\noindent\textbf{\Large{#1}}}

\newcommand{\mybib}[3]{\item #1: \textit{#2.} #3.}
\newcommand{\mybibhl}[3]{\item #1: \textit{#2.} \\ #3.}
\newcommand{\mythesis}[3]{#1: \textit{#2} & #3 \\ \addlinespace[1mm]}
\newcommand{\mythesisind}[4]{#1: \textit{#2} & #4 \\ 
	Industrial partner: #3\\ 	\addlinespace[1mm]}

\newcommand{\mypartheader}[1]{
	\newpage
	\phantomsection
\begin{paracol}{1} % Begin the multi-column environment
	\parbox[top][0.12\textheight][c]{\linewidth}{ 
		\vspace{-0.04\textheight}
		{\huge\color{headings} #1}
	}
\end{paracol}
\vspace{-4em}
\addcontentsline{toc}{chapter}{#1}
\lfoot{\textcolor{darkgray}{#1}}
}



\pagestyle{fancy} % Enable custom headers and footers

\fancyhf{} % This suppresses all headers and footers by default, add headers and footers in the template file as per the example





\begin{document}



%\renewcommand{\footrulewidth}{0.5pt}% default is 0pt
%\lfoot{\textcolor{darkgray}{Curriculum Vitae -- Prof.\ Dr.\ Han van der Aa}}
%\rfoot{\textcolor{darkgray}{\monthyeardate\today}}
%
%\makesavenoteenv{tabular}
%
	\newpage
\phantomsection
\addcontentsline{toc}{chapter}{Curriculum Vitae}
\lfoot{\textcolor{darkgray}{Curriculum Vitae -- Han van der Aa}}

\begin{paracol}{1} % Begin the multi-column environment



\parbox[top][0.12\textheight][c]{\linewidth}{ % Parbox to hold the author name and CV text; fixed height to match the coloured box to the right, centred vertically and full line width
	\vspace{-0.04\textheight} % Reduce whitespace above the parbox to separate it from the main content
	\centering % Centre text
	{\sffamily\Huge Prof.\ Dr.\ Han van der Aa}\\\medskip % Your name
	{\huge\color{headings} Curriculum Vitae }
}
%\switchcolumn
\end{paracol}
\vspace{0.3em}
\begin{paracol}{2}
	\hspace{2em}
\parbox[top][0.12\textheight][c]{1.00\linewidth}{ % Parbox to hold the colour box; fixed height to match the name/CV text to the left, centred vertically and full line width
	\vspace{-0.04\textheight} % Reduce whitespace above the parbox to separate it from the main content
	\colorbox{white}{ % Create the coloured box
		\begin{supertabular}{ll} % Start a table with two columns, the table will ensure everything is aligned
			\raisebox{-1pt}{Full name} & Johannes Hendrikus van der Aa \\ %
			\raisebox{-1pt}{Date of Birth} & 9 October 1988 \\ %
			\raisebox{-1pt}{Place of Birth} & Sint-Oedenrode, the Netherlands \\ %
			\raisebox{-1pt}{Address} & Hafenstrasse 31, 68159, Mannheim, Germany \\ % Address
			\raisebox{-1pt}{Phone:} & +49 (1) 763 460 1322 \\ % Phone number
			\raisebox{-1pt}{E-mail} & \href{mailto:han@informatik.uni-mannheim.de}{han@informatik.uni-mannheim.de} \\ % Email address
			\raisebox{0pt}{Homepage} & \href{https://www.hanvanderaa.com}{www.hanvanderaa.com} \\ % 
		\end{supertabular}
	}
}
\switchcolumn
%\begin{figure}
\vspace{-1.5em}
\hspace{12em}
	\includegraphics[height=3.3cm]{figures/headshot.png}
%\end{figure}

\end{paracol}

\medskip
\vspace{0.3em}
\section{Work Experience}
\begin{tabular}{p{12.5cm}l}
		\normalsize \textbf{Junior Professor in Artificial Intelligence Methods} & 		04/2020 -- current\\
		Head of the research group on Process Analytics \\
			Funded by the \textit{Künstliche Intelligenz Baden-Württemberg} (KI-BW) initiative \\ 
 School of Business Informatics and Mathematics \\
 	University of Mannheim, Germany \\
	\noalign{\smallskip\smallskip}
	
	\normalsize \textbf{Alexander von Humboldt Fellow (post-doctoral researcher)} & 			05/2018  --  03/2020 \\
	Databases \& Information Systems Group, Department of Computer Science\\
		Humboldt-Universit\"at zu Berlin, Germany \\
	\noalign{\smallskip\smallskip}
	
	\normalsize \textbf{Post-doctoral Researcher} & 		01/2018  --  04/2018 \\
	Business Informatics Group, Department of Computer Sciences \\
		Vrije Universiteit Amsterdam, the Netherlands \\
	\noalign{\smallskip\smallskip}
	
	
	\normalsize \textbf{Ph.D. Candidate} & 05/2014  --  01/2018 \\
	Business Informatics Group, Department of Computer Sciences \\
	Vrije Universiteit Amsterdam, the Netherlands \\
	\noalign{\smallskip\smallskip}
	
	\normalsize \textbf{Junior Researcher} & 09/2013  --  02/2014 \\
	SAP AG, Germany \\
%	Joint research in collaboration with the Eindhoven University of Technology\\
	
\end{tabular}


\medskip
\vspace{0.3em}
\section{Education}
\begin{tabular}{p{12.5cm}l}
 \normalsize \textbf{Ph.D. in Computer Science } & 	05/2014  --  01/2018 \\
	 Vrije Universiteit Amsterdam, the Netherlands \\
	 Thesis: \textit{Comparing and Aligning Process Representations} \\
%	 Supervisors: Prof.\ Dr.\ Hajo A. Reijers and Dr.\ Henrik Leopold \\
	 Honors: \textit{cum laude} (top 5\% in the Netherlands) \\
	\noalign{\smallskip\smallskip}
	
	 \normalsize \textbf{M.Sc. in Business Information Systems } & 09/2010  --  04/2013\\
	 Eindhoven University of Technology, the Netherlands \\
	 Thesis: \textit{Composing Workflow Activities} \\
	Honors: \textit{cum laude} (top 5\% of the university) \\
	\noalign{\smallskip\smallskip}

	
	 \normalsize \textbf{B.Sc. in Industrial Engineering } & 09/2007  --  07/2010 \\
	 Eindhoven University of Technology, the Netherlands \\
	 Honors: \textit{cum laude} (top 5\% of the university)
		
\end{tabular}

\section{Research Visits}
\begin{tabular}{p{12.5cm} p{3cm}  }
	\textbf{Universit\'e Paris Dauphine, France -- Visiting professor} & 10/2021 -- current  \\
	LAMSADE Laboratory \\
	\noalign{\smallskip\smallskip}
	
	\textbf{Technion - Israel Institute of Technology, Israel -- Visiting researcher} & 11/2017 -- 12/2017 \\
	Faculty of Industrial Engineering \& Management \\
	%	 Work on instance-based process matching with Prof.\ Dr.\ Avigdor Gal \\
	\noalign{\smallskip\smallskip}
	
	\textbf{Vienna Univ. of Economics and Business, Austria -- Visiting researcher} & 06/2013 -- 08/2013\\
	Department of Information Systems and Operations \\
	%	 Work on anomaly detection in flight trajectories with Prof.\ Dr.\ Jan Mendling \\
\end{tabular}


\section{Awards}
\begin{tabular}{lll  }
	\normalsize \textbf{Best student paper} &
	Int. conf. on Conceptual Modeling (ER) & 2020 \\
	
	\normalsize \textbf{Runner-up best paper} &
	Int. conf.  on Business Process Management (BPM) & 2019 \\
	\normalsize \textbf{Runner-up best thesis} &
	Int. conf. on Business Process Management (BPM) & 2018\\
	\normalsize \textbf{Runner-up best paper} & 
	Int. conf.  on Advanced Information Systems Engineering (CAiSE) & 2017\\
	
\end{tabular}

\medskip 

\section{Acquisition of Third-Party Funds}
\begin{tabular}{p{1.3cm}p{10.8cm}l}
	\multicolumn{2}{l}{\normalsize \textbf{Alexander von Humboldt Fellowship}} & 05/2018 -- 04/2020\\
	Agency: &Alexander von Humboldt Foundation \\
	Call: & Research Fellowships for post-doctoral researchers \\
	Proposal: & Temporal Uncertainty in Conformance Checking of Business Processes \\
	Role: & Principal Investigator  \\
	Amount: & \small\textbf{85,500 EUR} \\
	
\end{tabular}



\medskip
\section{Invited Talks }

	\textbf{Research Talks}
	\smallskip
	
\begin{tabular}{p{13.5cm}l}

	
\textbf{Challenges and Opportunities of Natural Language Processing in BPM} & 12/2019 \\	
{University of Seville, Spain}  \\
	\noalign{\smallskip\smallskip}

\textbf{Complex Event Processing for Event-Based Process Querying} (keynote) & 09/2019 \\
{International Workshop on Process Querying} \\
	\noalign{\smallskip\smallskip}

\textbf{Process Mining for Automated Compliance Checking} & 11/2017\\	
	{ING Bank, Amsterdam, the Netherlands}  \\

	\noalign{\smallskip\smallskip}
	
\textbf{Comparing Textual Process Descriptions to Process Models} & 01/2017\\
	{Technical University of Barcelona, Spain} \\

	\noalign{\smallskip\smallskip}

	\textbf{An introduction to Process Model Matching} (guest lecture) & 11/2016 \\	
	{Technion -- Israel Institute of Technology, Haifa, Israel}  \\

	\noalign{\smallskip\smallskip}
\textbf{Detecting Inconsistencies between Process Models and Textual Descriptions} & 11/2014 \\
	{Vienna University of Economics and Business, Austria} \\

	
\end{tabular}

\medskip
\textbf{Tutorials} 
\smallskip 
 
\begin{tabular}{p{13.5cm}l}
	
	
	\textbf{RuM: Declarative Process Mining, Distilled} & 09/2021  \\
%	With Anti Alman, Claudio di Ciccio, Marco Montali, and Fabrizio M. Maggi \\
	Int. Conf. on Business Process Management \\
	\noalign{\smallskip\smallskip}
	
		\textbf{Process Mining: Leveraging Event Data to Understand and Improve Organizations} & 12/2019 \\
%	With Henrik Leopold \\
	IEEE Int. Conf. on Big Data \\
	\noalign{\smallskip\smallskip}

\end{tabular}


\medskip
\section[Scientific Service]{Scientific Service (Selection)}


\textbf{Track Chair} 
\smallskip 

\begin{tabular}{p{13.5cm}l}
%	\textbf{Track Chair} \\
	HICSS Mini-Track on Business Process Technology & \hphantom{2020 - }2022 \\
		\noalign{\smallskip\smallskip}
\end{tabular}

\textbf{Program Committees}

\begin{tabular}{p{13.5cm}l}
	
		Int. conf. on Process Mining (ICPM) & \hphantom{2020 - }2021 \\
			Int. conf on Distributed and Event-Based Systems (DEBS) & \hphantom{2020 - }2021  \\
	Int. conf. on Business Process Management (BPM) & 2020 - 2021 \\
Int. working conf. on Business Process Modeling, Development, and Support (BPMDS) & 2019 -- 2021 \\
		International Joint Conference on Neural Networks (IJCNN) & \hphantom{2020 - }2020 \\
		
	\noalign{\smallskip\smallskip}
Workshop on Process Querying & 2019 -- 2021\\

		Workshop on Artificial Intelligence for Business Process Management  & 2018 -- 2021\\
Workshop on Declarative/Decision/Hybrid Mining and Modelling for Business Processes & 2018 -- 2021\\
Workshop on Blockchains for Inter-Organizational Collaboration & \hphantom{2020 - }2018\\
Workshop on Cognitive Business Process Management & \hphantom{2020 - }2017 \\
\end{tabular}

\newpage 
\textbf{Reviewer for Journals}
\smallskip 

\begin{tabular}{p{1.7cm}p{10.5cm}l}
	 	BISE & Business \& Information Systems Engineering \\
	 	COMIND & Computers in Industry \\
	 	COMP & Computing \\ 
 	 	DSS &  	Decision Support Systems \\
 	 	ESWA & Expert Systems with Applications \\ 
	 	IS & Information Systems \\
	 	JODS & Journal on Data Semantics \\
	 	SoSym & Software and Systems Modeling \\
	 	TKDE & IEEE Transactions on Knowledge and Data Engineering \\
\end{tabular}

\smallskip 
\textbf{Reviewer for Conferences}
\smallskip	

\begin{tabular}{p{1.7cm}p{11.3cm}l}
	
HICSS &	Hawaii International Conference on System Sciences &  \hphantom{2018 -- }2021 \\
	ECIS &	European Conference on Information Systems&  2018 -- 2021 \\
	AMCIS & 	Americas Conference on Information Systems &  \hphantom{2018 -- }2017 \\
		BPMDS & Business Process Modeling, Development, and Support &  2015 --  2017 \\
%		SAC &  ACM Symposium on Applied Computing & 2017 \\
		WI & International Conference on Wirtschaftsinformatik & \hphantom{2018 -- }2017 \\
		ICSOC & International Conference on Service Oriented Computing &  2015 -- 2016 \\
%		BPM & International Conference on Business Process Management &  2016 \\
\end{tabular}

\smallskip
\textbf{Reviewer for Funding Agencies}
\smallskip

\begin{tabular}{p{1.7cm}p{11.3cm}l}
SNF & Swiss National Science Foundation \\
\end{tabular}




%\mypartheader{Publications}
\section{Publications}
\noindent My work has been published in several prestigious journals and international conferences, including: 
\begin{itemize}
	\itemsep-0.2em
	
	\item Information Systems  (IS) -- 6x
	
	\item Decision support systems (DSS) -- 2x
	
	\item IEEE Transactions on Knowledge and Data Engineering (TKDE) -- 1x
	
	\item Data and Knowledge Engineering  (DKE) -- 2x
	
	\item Int.\ Conf.\ on Advanced Information Systems Engineering (CAISE)	 -- 7x
	
	\item Int.\ Conf.\ on Business Process Management (BPM) --  5x

	\item Int.\ Conf.\ on Computational Linguistics (COLING) -- 1x
	
	\item ACM Int.\ Conf.\ on Management of Data (SIGMOD) -- 1x 
	
	
\end{itemize}
\smallskip
Pre-print versions of all my publications are available at \href{https://www.hanvanderaa.com/publications}{www.hanvanderaa.com/publications}\\

%\section{Bibliographic Profile}
%\begin{figure}[!h]
%	\centering
%	\includegraphics[width=0.38\linewidth]{figures/scholarprofile.png}
%\end{figure}
%
%Updated:  June 29, 2021 (source: \href{https://scholar.google.com/citations?user=oFagE6cAAAAJ}{scholar.google.com/citations?user=oFagE6cAAAAJ})\\

\section{Selected Publications}

\begin{enumerate}[label=\arabic*.]
	\itemsep-0.3em
	\mybib{\textbf{Han van der Aa}, Adrian Rebmann, Henrik Leopold}
	{Natural language-based detection of semantic execution anomalies in event logs}
	{Information System 102: 101824, 2021}
	
			\mybib{ Bo Zhao, \textbf{Han van der Aa}, Nguyen Thanh Tam, Nguyen Quoc Viet Hung, Matthias Weidlich} {EIRES Efficient Integration of Remote Data in Event Stream Processing}
	{ACM SIGMOD International Conference on Management of Data (SIGMOD 2021)}
	
	\mybib{ Adrian Rebmann, \textbf{Han van der Aa}} {Extracting Semantic Process Information from the Natural Language in Event Logs} 
	{International Conference on Advanced Information Systems Engineering (CAiSE 2021)}
	
	\mybib{Martin Bauer, \textbf{Han van der Aa}, Matthias Weidlich}
	{Sampling and Approximation Techniques for Efficient Process Conformance Checking}
	{Information Systems 2020 (accepted for publication)}
	
	\mybib{\textbf{Han van der Aa}, Henrik Leopold, Matthias Weidlich}
	{Partial Order Resolution of Event Logs for Process Conformance Checking}
	{Decision Support Systems 136: 113347, 2020}



	\mybib{\textbf{Han van der Aa}, Henrik Leopold, Hajo A. Reijers}{Efficient Process Conformance Checking on the Basis of Uncertain Event-to-Activity Mappings}
{IEEE Transactions on Knowledge and Data Engineering  32(5): 927-940, 2020}

	\mybib{ Stephan A Fahrenkrog-Petersen, \textbf{Han van der Aa}, Matthias Weidlich} {	PRIPEL: Privacy-Preserving Event Log Publishing Including Contextual Information}
{International Conference on Business Process Management (BPM 2020)}

	\mybib{ \textbf{Han van der Aa}, Claudio Di Ciccio, Henrik Leopold and Hajo A Reijers} {Extracting Declarative Process Models from Natural Language} 
{International Conference on Advanced Information Systems Engineering (CAiSE 2019)}
	
\mybib{\textbf{Han van der Aa}, Henrik Leopold, Hajo A. Reijers}{Comparing Textual Descriptions to Process Models – The Automatic Detection of Inconsistencies}
{Information Systems 64: 447-460, 2017}

\mybib{ Claudio Di Ciccio, \textbf{Han van der Aa}, Cristina Cabanillas, Jan Mendling, Johannes Prescher}{Detecting Flight Trajectory Anomalies and Predicting Diversions in Freight Transportation}
{Decision Support Systems 88(1): 1-17, 2016}

\end{enumerate}
%\section{Journal Articles}


%\begin{enumerate}[label=J\arabic*.]
%	\itemsep-0.3em
%	\mybib{\textbf{Han van der Aa}, Adrian Rebmann, Henrik Leopold}
%	{Natural language-based detection of semantic execution anomalies in event logs}
%	{Information System 102: 101824, 2021}
%	
%	
%	\mybib{Martin Bauer, \textbf{Han van der Aa}, Matthias Weidlich}
%	{Sampling and Approximation Techniques for Efficient Process Conformance Checking}
%	{Information Systems 2020 (accepted for publication)}
%	
%	\mybib{\textbf{Han van der Aa}, Henrik Leopold, Matthias Weidlich}
%	{Partial Order Resolution of Event Logs for Process Conformance Checking}
%	{Decision Support Systems 136: 113347, 2020}
%
%		\mybib{Nimrod Busany, \textbf{Han van der Aa}, Arik Senderovich, Avigdor Gal, Matthias Weidlich}
%{Interval-based Queries over Lossy IoT Event Streams}
%{ACM Transactions on Data Science 1(4): 27:1-27:27, 2020}
%	
%	\mybib{\textbf{Han van der Aa}, Henrik Leopold, Hajo A. Reijers}{Efficient Process Conformance Checking on the Basis of Uncertain Event-to-Activity Mappings}
%	{IEEE Transactions on Knowledge and Data Engineering  32(5): 927-940, 2020}
%	
%	
%	
%	\mybib{ Henrik Leopold, \textbf{Han van der Aa}, Jelmer Offenberg, Hajo A. Reijers}{Using Hidden Markov Models for the Accurate Linguistic Analysis of Process Model Activity Labels} {Information Systems 83: 30-39, 2019}
%	
%	\mybib{ Henrik Leopold, \textbf{Han van der Aa}, Fabian Pittke, Manuel Raffel, Jan Mendling, Hajo A. Reijers}{Searching Textual and Model-based Process Descriptions based on a Unified Data Format}
%	{Software and Systems Modeling, 18(2):1179-1194, 2019}
%	
%	\mybib{ Josep S\'anchez-Ferreres, \textbf{Han van der Aa}, Josep Carmona, Llu\'is Padro}{Aligning Textual and Model-Based Process Descriptions}
%	{Data \& Knowledge Engineering, 118:24-40, 2018}
%	
%	\mybib{ Elena Kuss, Henrik Leopold, \textbf{Han van der Aa}, Heiner Stuckenschmidt, Hajo A. Reijers} {A Probabilistic Evaluation Procedure for Process Model Matching Techniques}
%	{Data \& Knowledge Engineering, 117:393-406, 2018}
%	
%	\mybib{ \textbf{Han van der Aa}, Henrik Leopold, Hajo A. Reijers}{Checking Process Compliance against Natural Language Specifications using Behavioral Spaces}
%	{Information Systems, 78:83-95, 2018}
%	
%	
%	\mybib{ \textbf{Han van der Aa}, Henrik Leopold, Adela del-Rio-Ortega, Manuel Resinas, Hajo A. Reijers}
%	{Transforming Unstructured Natural Language Descriptions into Measurable Process Performance Indicators Using Hidden Markov Models}
%	{Information Systems 71:27-39, 2017}
%	
%	\mybib{\textbf{Han van der Aa}, Henrik Leopold, Hajo A. Reijers}{Comparing Textual Descriptions to Process Models – The Automatic Detection of Inconsistencies}
%	{Information Systems 64: 447-460, 2017}
%	
%	\mybib{ Claudio Di Ciccio, \textbf{Han van der Aa}, Cristina Cabanillas, Jan Mendling, Johannes Prescher}{Detecting Flight Trajectory Anomalies and Predicting Diversions in Freight Transportation}
%	{Decision Support Systems 88(1): 1-17, 2016}
%	
%	\mybib{ \textbf{Han van der Aa}, Hajo A. Reijers, Irene Vanderfeesten}{Designing Like a Pro: The Automated Composition of Workflow Activities}
%	{Computers in Industry 75(1): 162-177, 2016}
%	
%	
%\end{enumerate}
%
%\section{Conference Proceedings}
%
%\begin{enumerate}[label=C\arabic*.]
%	\itemsep-0.3em
%	
%		\mybib{ Bo Zhao, \textbf{Han van der Aa}, Nguyen Thanh Tam, Nguyen Quoc Viet Hung, Matthias Weidlich} {EIRES Efficient Integration of Remote Data in Event Stream Processing}
%	{ACM SIGMOD International Conference on Management of Data (SIGMOD 2021)}
%	
%	\mybib{ Adrian Rebmann, \textbf{Han van der Aa}} {Extracting Semantic Process Information from the Natural Language in Event Logs} 
%	{33rd International Conference on Advanced Information Systems Engineering (CAiSE 2021)}
%		
%	\mybib{ Bernhard Sch\"afer, \textbf{Han van der Aa}, Henrik Leopold, Heiner Stuckenschmidt} {Sketch2BPMN: Automatic Recognition of Hand-drawn BPMN Models}
%	{33rd International Conference on Advanced Information Systems Engineering (CAiSE 2021)}
%
%	\mybib{ Diana Sola, Christian Meilicke, \textbf{Han van der Aa}, Heiner Stuckenschmidt} {A Rule-based Recommendation Approach for Business Process Modeling}
%	{33rd International Conference on Advanced Information Systems Engineering (CAiSE 2021)}
%	
%	\mybib{ 	Karolin Winter, \textbf{Han van der Aa}, Stefanie Rinderle-Ma, Matthias Weidlich} {	Assessing the Compliance of Business Process Models with Regulatory Documents}
%	{39th International Conference on Conceptual Modeling (ER 2020)}
%	
%	\mybib{ Stephan A Fahrenkrog-Petersen, \textbf{Han van der Aa}, Matthias Weidlich} {	PRIPEL: Privacy-Preserving Event Log Publishing Including Contextual Information}
%	{18th International Conference on Business Process Management (BPM 2020)}
%	
%	\mybib{ 	Martin Bauer, \textbf{Han van der Aa}, Matthias Weidlich} {	Estimating Process Conformance by Trace Sampling and Result Approximation}
%	{17th International Conference on Business Process Management (BPM 2019)}
%	
%	\mybib{ Stephan A Fahrenkrog-Petersen, \textbf{Han van der Aa} and Matthias Weidlich} {PRETSA: Event Log Sanitization for Privacy-aware Process Discovery} 
%	{1st International Conference on Process Mining (ICPM 2019)}
%	
%	\mybib{ \textbf{Han van der Aa}, Claudio Di Ciccio, Henrik Leopold and Hajo A Reijers} {Extracting Declarative Process Models from Natural Language} 
%	{31st International Conference on Advanced Information Systems Engineering (CAiSE 2019)}
%	
%	\mybib{  \textbf{Han van der Aa}, Josep Carmona, Henrik Leopold, Jan Mendling, Lluis Padro} {Challenges and Opportunities of Applying Natural Language Processing in Business Process Management} 
%	{27th International Conference on Computational Linguistics (COLING 2018)}
%	
%	\mybib{ \textbf{Han van der Aa}, Henrik Leopold, Inge van de Weerd, Hajo A Reijers} {Causes and Consequences of Fragmented Process Information: Insights from a Case Study} 
%	{23rd Americas Conference on Information Systems (AMCIS 2017)}
%	
%	\mybib{ \textbf{Han van der Aa}, Avigdor Gal, Henrik Leopold, Hajo A Reijers, Tomer Sagi, Roee Shraga} {Instance-Based Process Matching using Event-Log Information} 
%	{29th International Conference on Advanced Information Systems Engineering (CAiSE 2017)}
%	
%	\mybib{ \textbf{Han van der Aa}, Henrik Leopold, Hajo A Reijers} {Checking Process Compliance on the Basis of Uncertain Event-to-Activity Mappings} 
%	{29th International Conference on Advanced Information Systems Engineering (CAiSE 2017)}
%	
%	\mybib{ Elena Kuss, Henrik Leopold, \textbf{Han van der Aa}, Heiner Stuckenschmidt, Hajo A Reijers} {Probabilistic Evaluation of Process Model Matching Techniques} 
%	{35th International Conference on Conceptual Modeling (ER 2016)}
%	
%	\mybib{ \textbf{Han van der Aa}, Henrik Leopold, Hajo A Reijers} {Dealing with Behavioral Ambiguity in Textual Process Descriptions} 
%	{14th International Conference on Business Process Management (BPM 2016)}
%	
%	\mybib{ Ermeson Andrade, \textbf{Han van der Aa}, Henrik Leopold, Steven Alter, Hajo A Reijers} {Factors Leading to Business Process Noncompliance and its Positive and Negative Effects: Empirical Insights from a Case Study}
%	{22nd Americas Conference on Information Systems (AMCIS 2016)}
%	
%	\mybib{ \textbf{Han van der Aa}, Adela del-Rio-Ortega, Manuel Resinas, Henrik Leopold, Antonio Ruiz-Cortés, Jan Mendling, Hajo A Reijers} 
%	{Narrowing the Business-IT Gap in Process Performance Measurement} 
%	{28th International Conference on Advanced Information Systems Engineering (CAiSE 2016)}
%	
%	\mybib{ \textbf{Han van der Aa}, Henrik Leopold, Hajo A Reijers} {Detecting Inconsistencies between Process Models and Textual Descriptions} 
%	{13th International Conference on Business Process Management (BPM 2015)}
%	
%	\mybib{ \textbf{Han van der Aa}, Hajo A Reijers, Irene Vanderfeesten} {Composing Workflow Activities on the Basis of Data-flow Structures} 
%	{11th International Conference on Business Process Management (BPM 2013)}
%\end{enumerate}
%
%\section{Book}
%
%\begin{enumerate}[label=B\arabic*.]
%	
%	\mybib{\textbf{Han van der Aa}}
%	{Comparing and Aligning Process Representations: Foundations and Technical Solutions} {Lecture Notes on Business Information Processing (LNBIP), Vol. 323, Springer-Verlag, 2018}
%	
%\end{enumerate}
%
%\section{Book Chapters}
%
%\begin{enumerate}[label=Ch\arabic*.]
%	\itemsep-0.2em
%	\mybib{Henrik Leopold and \textbf{Han van der Aa} }
%	{Automatically Identifying Process Automation Candidates Using Natural Language Processing}
%	{Blockchains \& RPA, Springer (forthcoming, 2021)}
%	
%	
%	\mybib{\textbf{Han van der Aa} and Henrik Leopold}
%	{Supporting Robotic Process Automation through Natural Language Processing}
%	{Robotic Process Automation, De Gruyter STEM (2021)}
%	
%	\mybib{\textbf{Han van der Aa}, Alexander Artikis, Matthias Weidlich}
%	{Complex Event Processing Methods for Process Querying}
%	{Process Querying Methods (forthcoming, 2021)}
%	
%\end{enumerate}
%
%\section{Workshop and Working Conference Proceedings}
%\begin{enumerate}[label=W\arabic*.]
%	\itemsep-0.3em
%	
%	\mybib{ Anti Alman, Karl Johannes Balder, Fabrizio M. Maggi, \textbf{Han van der Aa}}
%{Declo: A Chatbot for User-friendly Specification of Declarative Process Models}
%	{18th International conference on Business Process Management (BPM Demos 2020)}
%	
%	\mybib{\textbf{Han van der Aa}, Karl Johannes Balder, Fabrizio M. Maggi, Alexander Nolte}
%{Say It In Your Own Words: Defining Declarative Process Models Using Speech Recognition}
%	{18th International conference on Business Process Management (BPM Forum 2020)}
%	
%	\mybib{	Martin Bauer, Stephan A. Fahrenkrog-Petersen, Agnes Koschmider, Felix Mannhardt, \textbf{Han van der Aa}, Matthias Weidlich} 
%	{ELPaaS: Event Log Privacy as a Service}
%	{17th International Conference on Business Process Management (BPM Demos 2019)}
%
%	\mybib{ Jan Mendling, Henrik Leopold, Lucinéia Heloisa Thom, \textbf{Han van der Aa}}
%	{Natural Language Processing with Process Models} 
%	{2nd Workshop on Natural Language Processing for Requirements Engineering \& NLP}
%%
%	\mybib{ Michael Offel, \textbf{Han van der Aa}, Matthias Weidlich}
%	{Towards Net-based Formal Methods for Complex Event Processing}
%	{Large-scale Data Management and Processing – Applications in Research and Industry (LWDA 2018)}
%	
%	\mybib{Henrik Leopold, \textbf{Han van der Aa}, Hajo A. Reijers}
%	{Identifying Candidate Tasks for Robotic Process Automation in Textual Process Descriptions} {BPMDS’18 Working Conference (BPMDS 2018)}
%	
%	\mybib{ Henrik Leopold, \textbf{Han van der Aa}, Fabian Pittke, Manuel Raffel, Jan Mendling, Hajo A. Reijers}
%	{Integrating Textual and Model-based Process Descriptions for Comprehensive Process Search} {BPMDS’16 Working Conference (BPMDS 2016)}
%	
%	\mybib{ \textbf{Han van der Aa}, Henrik Leopold, Kimon Batoulis, Mathias Weske, Hajo A. Reijers} {Integrated Process and Decision Modeling for Data-Driven Processes}
%	{3th International Workshop on Decision Mining \& Modeling for Business Processes (DeMiMoP’15)}
%	
%	\mybib{ \textbf{Han van der Aa}, Henrik Leopold, Felix Mannhardt, Hajo A. Reijers}
%	{On the Fragmentation of Process Information: Challenges, Solutions, and Outlook}
%	{BPMDS’15 Working Conference (BPMDS 2015)}
%	
%\end{enumerate}




\end{document}